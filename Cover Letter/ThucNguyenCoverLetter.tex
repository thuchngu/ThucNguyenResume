\documentclass{article}
\usepackage[T1]{fontenc}
\usepackage{stix}
\usepackage{nopageno}
\usepackage{titlesec}
\usepackage{titling}
\usepackage[margin=.5in]{geometry}

\titleformat{\section}
{\large\bfseries}
{}
{0em}
{}[\titlerule]
\titleformat
{\subsection}[runin]
{\medium\bfseries}
{}
{0em}
{}
\renewcommand{\maketitle}
{\begin{center}
{\huge\bfseries\theauthor}
\vspace{.5em}

thuchngu@bu.edu

https://www.linkedin.com/in/thucnguyen61098 --- https://github.com/thuchgnu

\end{center}}

\begin{document}
\title{Resume}
\author{Thuc Nguyen}
\maketitle
\vspace{-.5em}
{\centering
17 Aberdeen St. Apt 6 Boston, MA 02215

845-416-8037

}

\vspace{2em}
August 21, 2019

\vspace{2em}

Ms. Kelly Ryan

Vice President, IBM Z Support

IBM

2455 South Rd.

Poughkeepsie, NY 12601
\vspace{2em}

Ms. Kelly Ryan,

\vspace{1em}
After 3 years of undergraduate education and realizing that computer hardware and digital design are my passions, I am looking towards life after my Bachelor's Degree and think that IBM is where I would like to begin pursuing those passions professionally as a Hardware Developer.

\vspace{1em}
As a college undergraduate entering my 4th and final year of education, I've learned a number of different skills that I could build upon as a hardware developer at IBM. Verilog is my strongest hardware description language and C++, Java, and Python are the other programming languages I have worked with. I also have a basic working knowledge of KiCAD as I'm currently working on a PCB design project. Because I am still an undergraduate, I do not have much professional experience. However, I am ready and willing to learn the skills needed to succeed at IBM. The skills I can offer are:

\begin{itemize}
  \item{Logic Design using Verilog}
  \item{Verilog Design Simulation using Xilinx ISE Design Suite}
  \item{Schematic and PCB Design using KiCAD and EAGLE}
  \item{Object-Oriented Programming in C++, Java, and Python}
  \item{Working Knowledge of Programming in Arduino}
  \item{Version Control using Git/Github}
\end{itemize}

\vspace{1em}
In my current position as a research assistant in Boston University's Space Technology Lab working on the ANDESITE project, I've used Atmel Studio to program and debug the Arduino chips on our node satellites, EAGLE to inspect the schematics and PCBs of the boards on our main and node satellites as well as create BOMs for those boards, as well as PyCharm to test and debug the main satellite's communcation with our ground station. 
\end{document}
