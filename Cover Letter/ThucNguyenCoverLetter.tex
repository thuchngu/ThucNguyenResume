\documentclass{article}
\usepackage[T1]{fontenc}
\usepackage{stix}
\usepackage{nopageno}
\usepackage{titlesec}
\usepackage{titling}
\usepackage[margin=1in]{geometry}

\titleformat{\section}
{\large\bfseries}
{}
{0em}
{}[\titlerule]
\titleformat
{\subsection}[runin]
{\medium\bfseries}
{}
{0em}
{}
\renewcommand{\maketitle}
{\begin{center}
{\huge\bfseries\theauthor}
\vspace{.5em}

https://www.linkedin.com/in/thucnguyen61098 --- https://github.com/thuchgnu

\end{center}}

\begin{document}
\title{Resume}
\author{Thuc Nguyen}
\maketitle
\vspace{-.5em}

\begin{flushleft}
\vspace{1.6em}

September 30, 2019

\vspace{1.6em}
Ekomobong Uko

Lutron Electronics, Inc.

1 Beacon St.

Boston, MA 02108

\end{flushleft}
\vspace{1.6em}
Ekomobong Uko,

\vspace{1em}
After 3 years of undergraduate education and realizing that computer hardware and digital design are my passions, I am looking towards life after my Bachelor's Degree and think that Lutron is where I would like to begin pursuing those passions professionally as an embedded engineer.

\vspace{.6em}
After attending Lutron's info session at Boston University and learning about the products that the company develops, I immediately identified with the "tinkering" personality trait that Lutron looks for in its engineers. I've learned a number of different skills through coursework and in my own personal time that I could build upon as an embedded engineer at Lutron and I'm eager to learn any new skills I would need to succeed at Lutron. I have experience working with C/C++, Java, and Python for higher level programming, as well as Verilog for digital design. I also have a basic working knowledge of KiCAD and EAGLE as I am currently working on a PCB design project. Some of the specific skills I can offer are:

\begin{itemize}
  \item{Logic Design using Verilog on an FPGA}
  \item{Verilog Design Simulation using Xilinx ISE Design Suite}
  \item{Schematic and PCB Design using KiCAD and EAGLE}
  \item{Peripheral Interfacing with GPIO, I2C, and UART}
  \item{Object-Oriented Programming in C/C++, Java, and Python}
  \item{Working Knowledge of Programming in Arduino}
  \item{Version Control using Git/Github}
\end{itemize}

\vspace{.6em}
In my current position as a research assistant  working on the ANDESITE project in Boston University's Space Technology Lab, I've used Atmel Studio to program and debug the Arduino chips on our node satellites, EAGLE to inspect the schematics and PCBs of the boards on our main and node satellites, as well as PyCharm and CLion to test and debug the main satellite's Globalstar communication with our ground station and RF communication with the node satellites.

\vspace{.6em}
Thank you for taking the time to consider me as a potential candidate and I hope we can meet in person so that we can start a conversation about how I can contribute to Lutron as an engineer.

\begin{flushleft}
\vspace{1.6em}
Best Regards,

\vspace{1em}

Thuc Nguyen

73 Thatcher St. Apt 1

Brookline, MA 02446

845-416-8037

thuchngu@bu.edu

\end{flushleft}
\end{document}
