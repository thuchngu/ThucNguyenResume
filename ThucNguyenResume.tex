\documentclass{article}
\usepackage[T1]{fontenc}
\usepackage{stix}
\usepackage{nopageno}
\usepackage{titlesec}
\usepackage{titling}
\usepackage[margin=.5in]{geometry}

\titleformat{\section}
{\large\bfseries}
{}
{0em}
{}[\titlerule]
\titleformat
{\subsection}[runin]
{\medium\bfseries}
{}
{0em}
{}
\renewcommand{\maketitle}
{\begin{center}
{\huge\bfseries\theauthor}
\vspace{.5em}

thuchngu@bu.edu

https://www.linkedin.com/in/thucnguyen61098 --- https://github.com/thuchgnu

\end{center}}

\begin{document}
\title{Resume}
\author{Thuc Nguyen}
\maketitle
\vspace{-.5em}
{\centering
73 Thatcher St. Apt 1 Brookline, MA 02446

845-416-8037

}
\vspace{-1.5em}
\section{Technical Skills}
\subsection{Proficient}
C, C++, Java, Arduino, Verilog, MIPS, Xilinx ISE Design Suite, Git, KiCAD, Linux/UNIX, Soldering
\vspace{-.75em}
\subsection{Familiar}
SPICE, EAGLE, Atmel Studio, Python, Android Studio, MATLAB, LaTex

\vspace{-.75em}
\section{Work Experience}
\subsection{Boston University Center for Space Physics}  \textit{Research Assistant - ANDESITE Software Team}
May 2018 - Present

Test and debug RF communication between node satellites and mule satellite using Arduino and Atmel Studio. Utilize NSL groundstation web service and Globalstar Satellite network to test ground-to-mule communication and data handling in outdoor testing on engineering model satellite. Assemble and reflow new flight-spare node PCBs.
\vspace{-.75em}
\subsection{Mugar Memorial Library} \textit{Library Assistant}
September 2016 - December 2017

Check-in and check-out interlibrary loan items. Locate prepare items to be shipped to other libraries.
\vspace{-.75em}
\subsection{Town of Esopus Library} \textit{Library Page}
June 2013 - March 2016

Place returned items on the shelf, pull on-hold items off of the shelf, check-in and check-out items. Assist in opening and closing the library at the beginning and end of the day. Help patrons in person and over the phone to find items, place holds, sign up for events, and other general information inquiries. Organize and clean shelves, locate lost items, update seasonal items.

\vspace{-.75em}
\section{Education}
\textbf{Boston University} - \textit{Bachelor of Science in Computer Engineering}

September 2016 - May 2020(anticipated)

GPA: 3.20/4.00

\vspace{-.75em}
\section{Relevant Coursework}
{\centering
Computer Architecture - Computer Organization - Introduction to Logic Design - Advanced Data Structures - Applied Algorithms - Introduction to Software Engineering - Introduction to Engineering Computation - Introduction to Electronics - Electric Circuits - Introduction to Engineering Design
\par
}

\vspace{-.75em}
\section{Projects}
\subsection{Custom Mechanical Keyboard} \textit{Personal Project}

A Work-in-progress mechanical keyboard PCB designed from scratch in KiCAD with a custom compact key layout designed in keyboard-layout-editor. The layout consists of the main alphabet section of a keyboard without the function row and the number row. It also has an isolated navigation cluster and a simple 10 key number pad.
\vspace{-.75em}
\subsection{Simple Pipelined MIPS CPU} \textif{Final Project for Computer Orgainzation Course}

A simple MIPS CPU that utilizes pipelining and is programmed using Verilog in XILINX ISE Design Suite. Contains a register file, an ALU, instruction memory, data memory, hazard detection and handling, and forwarding.
\vspace{-.75em}
\subsection{Lamp Post Mounted Flood Detector} \textit{Final Project for Engineering Design Course}

A water level detector composed of an arduino, an ultrasonic distance sensor, an XBee radio module, and a float switch. The system outputs the water level to a hypothetical relay station and then outputs a warning message once the level has reached or surpassed 1 ft.
\vspace{-.75em}
\subsection{Verilog Digital Lock} \textit{Final Project for Logic Design Course}

A digital lock that utilizes a seven-segment display and a series of switches on an FPGA board that allow a user to input a password to unlock the lock, change the password, and lock the digital lock. The lock was programmed in Verilog and simulated using Xilinx ISE Design Suite.
\vspace{-.75em}
\subsection{Currency Conversion App} \textit{Final Project for Software Engineering Course}

An Android application developed in Android Studio as part of a team project. The app allows the user to input a source currency and an amount, and after tapping the "convert" button, displays the equivalent amount in several other world currencies.

\end{document}
