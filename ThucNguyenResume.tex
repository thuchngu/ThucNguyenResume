\documentclass{article}
\usepackage{nopageno}
\usepackage{titlesec}
\usepackage{titling}
\usepackage[margin=.5in]{geometry}

\titleformat{\section}
{\large\bfseries}
{}
{0em}
{}[\titlerule]
\titleformat
{\subsection}[runin]
{\medium\bfseries}
{$\bullet$}
{0em}
{}
\renewcommand{\maketitle}
{\begin{center}
{\huge\bfseries\theauthor}
\vspace{.5em}

thuchngu@bu.edu

https://www.linkedin.com/in/thucnguyen61098

https://github.com/thuchgnu

\end{center}}

\begin{document}
\title{Resume}
\author{Thuc Nguyen}
\maketitle
\vspace{-.5em}
{\centering
17 Aberdeen St. Apt 6

Boston, MA 02215

845-416-8037

}

\vspace{.5em}






\section{Technical Skills}
\subsection{Programming}
C, C++, Java, Python, Arduino, Verilog, MIPS, LaTex
\subsection{Other}
Linux/UNIX, Git, EAGLE, KiCAD, Atmel Studio, Xilinx ISE Design Suite

\section{Work Experience}
\subsection{Boston University Center for Space Physics}  \textit{Research Assistant - ANDESITE Software Team}
May 2018 - Present

Test and debug satellite code, analyze PCB schematics, test main sensors and electrical hardware on satellite
\subsection{Mugar Memorial Library} \textit{Library Assistant}
September 2016 - December 2017

Check-in and check-out interlibrary loan items, prepare items to be shipped to other libraries
\subsection{Town of Esopus Library} \textit{Library Page}
June 2013 - March 2016

Place returned items on the shelf, pull on-hold items off of the shelf, check-in and check-out items

\section{Education}
Boston University - \textit{Bachelor of Science in Computer Engineering}

September 2016 - May 2020(anticipated)

GPA: 3.20/4.00

\section{Projects}
\subsection{Custom Mechanical Keyboard} \textit{Personal Project}

A mechanical keyboard PCB designed from scratch in KiCAD with a custom layout designed in keyboard-layout-editor(project is still in progress).
\subsection{Lamp Post Mounted Flood Detector} \textit{Final Project for Engineering Design Course}

A water level detector composed of an arduino, an ultrasonic distance sensor, an XBee radio module, and a float switch. The system outputs the water level to a hypothetical relay station and then outputs a warning message once the level has reached or surpassed 1 ft.
\subsection{Verilog Digital Lock} \textit{Final Project for Logic Design Course}

A digital lock that utilizes a seven-segment display and a series of switches on an FPGA board that allow a user to input a password to unlock the lock, change the password, and lock the digital lock. The lock was programmed in Verilog and simulated using Xilinx ISE Design Suite.

\end{document}
