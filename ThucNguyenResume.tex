\documentclass{article}
\usepackage[T1]{fontenc}
\usepackage{stix}
\usepackage{nopageno}
\usepackage{titlesec}
\usepackage{titling}
\usepackage[margin=.5in]{geometry}

\titleformat{\section}
{\large\bfseries}
{}
{0em}
{}[\titlerule]
\titleformat
{\subsection}[runin]
{\medium\bfseries}
{}
{0em}
{}
\renewcommand{\maketitle}
{\begin{center}
{\huge\bfseries\theauthor}
\vspace{.5em}

thuchngu@protonmail.com

https://www.linkedin.com/in/thucnguyen61098 --- https://github.com/thuchngu

\end{center}}

\begin{document}
\title{Resume}
\author{Thuc Nguyen}
\maketitle
\vspace{-.5em}
{\centering
223 Concord Turnpike \#308, Cambridge, MA 02140

845-416-8037

}
\vspace{-1.5em}
\section{Education}
\textbf{Boston University}  \textit{Bachelor of Science in Computer Engineering}

September 2016 - May 2020

GPA: 3.18/4.00

\vspace{-.75em}
\section{Technical Skills}
\subsection{Proficient}
Python, Shell, Bash, C++, C, Arduino, Git, KiCAD, Linux/UNIX, Jira, BitBucket, Confluence, Soldering
\vspace{-.75em}
\subsection{Familiar}
Java, Verilog, FPGA, PADS Designer, Xilinx Vivado, VHDL, GPIO, UART, I2C, SPICE, CUDA, EAGLE, Atmel Studio, Android Studio, MATLAB, LaTex

\vspace{-.75em}
\section{Work Experience}
\subsection{Raytheon BBN Technologies} \textit{Engineer I, Research}
  September 7 2021 - Present

Research Engineer in the INSERT DEPARTMENT Department of Raytheon BBN Technologies located in Cambridge, MA.
\vspace{-.75em}
\subsection{US Naval Research Laboratory} \textit{Electronics Engineer}
  August 3 2020 - August 27 2021

Electronics engineer in the Space Systems Engineering Division of the US Naval Research Lab in Washington DC. Design test circuits for aircraft avionics equipment. Write and debug scripts as part of hardware integration and testing efforts.
\vspace{-.75em}
\subsection{Boston University Center for Space Physics}  \textit{Research Assistant - ANDESITE Software Team}
  May 2018 - May 2020

Test and debug RF communication between node satellites and mule satellite using Arduino and Atmel Studio. Utilize NSL groundstation web service and Globalstar Satellite network to test ground-to-mule communication and data handling in outdoor testing on engineering model satellite. Assemble and reflow new flight-spare node PCBs.
\vspace{-.75em}
\subsection{Mugar Memorial Library} \textit{Library Assistant}
  September 2016 - December 2017

Check-in and check-out interlibrary loan items. Locate prepare items to be shipped to other libraries.
\vspace{-.75em}
\subsection{Town of Esopus Library} \textit{Library Page}
  June 2013 - March 2016

Place returned items on the shelf, pull on-hold items off of the shelf, check-in and check-out items. Assist in opening and closing the library at the beginning and end of the day. Help patrons in person and over the phone to find items, place holds, sign up for events, and other general information inquiries. Organize and clean shelves, locate lost items, update seasonal items.

\vspace{-.75em}
\section{Relevant Coursework}
{\centering
Digital VLSI Circuit Design - Intro to Embedded Systems - Intro to Electronics - Electric Circuits - High Performance Computing - Computer Architecture - Computer Organization - Intro to Logic Design - Advanced Data Structures - Applied Algorithms
\par
}

\vspace{-.75em}
\section{Projects}
\subsection{Lunar Atmospheric Cubesat Imaging System} \textit{Senior Design Project}

A prototype imaging system using COTS components designed to survive in lunar orbit inside a 3U cubesat form factor. The mission for the cubesat was to capture images of the lunar atmosphere with several levels of filtering in order to see the gradient of potassium and sodium particles in the lunar atmosphere.
\vspace{-.75em}
\subsection{Keyboard PCB} \textit{Personal Project}

A Work-in-progress mechanical keyboard PCB designed from scratch in KiCAD with a custom compact key layout designed in keyboard-layout-editor. The layout consists of the main alphabet section of a keyboard without the function row and the number row. It also has an isolated navigation cluster and a simple 10 key number pad.
\vspace{-.75em}
\subsection{Simple Pipelined MIPS CPU} \textit{Final Project for Computer Organization Course}

A simulated MIPS CPU that utilizes pipelining and is programmed using Verilog in XILINX ISE Design Suite. Contains a register file, an ALU, instruction memory, data memory, hazard detection and handling, and forwarding.
\vspace{-.75em}
\subsection{Lamp Post Mounted Flood Detector} \textit{Final Project for Engineering Design Course}

A water level detector composed of an arduino, an ultrasonic distance sensor, an XBee radio module, and a float switch. The system outputs the water level to a hypothetical relay station and then outputs a warning message once the level has reached or surpassed 1 ft.
\vspace{-.75em}

\end{document}
